\chapter{Contagem na vida real}

Nos últimos capítulos, vimos algumas soluções do problema de estimar a quantidade de elementos distintos de um fluxo de
dados. A solução mais usada atualmente é a estrutura de dados \hyperref[sec:loglog:hyperloglog]{HyperLogLog}. 

O Redis, que é um banco de dados de chave-valor salvo em memória, oferece uma implementação dessa estrutura de dados. 
Nessa estrutura, precisamos definir um valor de~$m$, que é um parâmetro relacionado à precisão das estimativas e ao 
consumo de memória. No Redis, esse valor é de $2^{14} = 16384$. Assim, o desvio padrão dessa implementação é de 
$1{,}04 \mathbin{/} \sqrt{m} = 1{,}04 \mathbin{/} \sqrt{16384} \approx 0{,}81\%$.

A função de hash utilizada na implementação do Redis produz saídas de 64 bits. Como $m = 14$, os valores que são 
considerados nos lotes do $\HLOG$ tem $50$ bits, e são necessários somente $6$ bits para guardar a informação de cada 
lote. Em vista disso, são armazenados 16 mil lotes que consomem 6 bits cada, e o consumo da estrutura fica em torno de 
12~KB.

As aplicações que utilizam alguma solução do problema da contagem distinta aproximada precisam geralmente, manter 
quantos elementos distintos apareceram em \textbf{vários} fluxos de dados. A consequência disso é que surge a 
necessidade de por exemplo, salvarmos uma estrutura~$\HLOG$ em um banco de dados para evitar termos todas as estruturas 
desse tipo em memória. Nesse sentido, o Programa~\ref{hloglog:code} está incompleto, pois em um primeiro momento, não 
existe uma forma fácil de guardarmos a classe~\texttt{HyperLogLog} em um banco de dados.

A implementação do Redis contorna esse problema separando o código dos métodos \texttt{adiciona} e \texttt{conta} do 
armazenamento dos dados dos lotes~\citep{HyperLogLogDetails}. Por conta disso, a estrutura~$\HLOG$ do Redis é somente 
uma sequência de bits que comprime a informação dos lotes. E com essa representação em bits, podemos guardar essa 
estrutura no banco de dados.

\section{Quantas visitas um site teve?}

Uma das métricas interessantes que aplicativos e sites podem coletar é a quantidade de acessos diários. A importância 
desse tipo de informação é que ela pode ajudar a responder perguntas como ``Tenho muitos acessos nos fins de semana?'' 
ou ``Depois do lançamento de uma certa funcionalidade, o número de acessos aumentou?''.

Contudo, pode existir situações em que é importante identificar o número de acessos distintos. Esse dado pode ser 
interessante para lojas virtuais ou redes sociais, uma vez que, essas aplicações dependem que várias pessoas diferentes
acessem esse tipo de site ou aplicativo.

Para resolvermos esse problema, podemos aproveitar o endereço de IP dos usuários que acessarem os servidores de um site.
Dessa forma, estaríamos trabalhando com um fluxo de endereços, e queremos descobrir quantos IP's distintos apareceram em
um certo dia. Portanto, inserimos os IP's que forem acessando o servidor em uma estrutura~$\HLOG$, e ao final do dia, 
salvamos a estimativa no banco de dados e zeramos o $\HLOG$. Assim, esse banco teria a informação dos acessos distintos 
diários. E essa ideia pode ser estendida para obtermos quantos acessos diferentes um site teve em uma certa hora do dia.

\section{Quantas pessoas leram um artigo?}

O problema anterior exigiu o uso de apenas uma estrutura~$\HLOG$. Porém, existem situações nas quais mais estruturas 
desse tipo precisam ser utilizadas. Um desses casos é contar quantas visualizações um dado artigo teve.

Reddit é uma plataforma cuja principal funcionalidade é permitir a criação de canais de discussões sobre diversos 
assuntos. Uma métrica interessante, portanto, é quantos usuários distintos visualizaram uma discussão específica. Apesar
de esse site não estar coletando atualmente essa métrica por conta de problemas de performace~\citep{RedditDisabled}, a 
solução apresentada pela equipe de engenharia dele exibe uma situação em que várias estruturas~$\HLOG$ são 
utilizadas~\citep{Reddit}.

Os engenheiros do Reddit utilizaram a implementação do~$\HLOG$ disponível no Redis. Para entendermos os próximos passos,
é necessário termos noção de como funciona esse banco de dados. O Redis é basicamente um conjunto de chaves e valores, e
o programador tem a sua disposição alguns métodos para interagir com esse conjunto. Um desses métodos é o \texttt{SET},
que recebe os parâmetros \texttt{key} e \texttt{value} e adiciona a essa coleção esse par de chave e valor. Se já 
existir um par com a chave \texttt{key}, o valor antigo é substituído por \texttt{value}.

Outro comando útil é o \texttt{PFADD}, que recebe uma chave~\texttt{key} e uma palavra. Este método insere elementos em 
uma estrutura~$\HLOG$ e retorna 1 se o item for de fato inserido ou 0, caso contrário. Nesse sentido, o Redis supõem que 
o valor da chave~\texttt{key} é uma sequência de bits que representa um $\HLOG$. Caso a chave~\texttt{key} não exista no 
momento da inserção, uma estrutura vazia é criada. E também existe o comando \texttt{PFCOUNT} que recebe uma chave 
\texttt{key} e retorna a estimativa para a quantidade de dados diferentes inseridos na estrutura~$\HLOG$, que está 
armazenada no valor desta chave.

Em vista disso, a solução do Reddit armazena várias estruturas~$\HLOG$ em uma instância do Redis, de modo que cada 
estrutura está associada a uma discussão. Contudo, o Redis é um banco de dados salvo em memória, e não é possível 
guardar todas as estruturas de uma vez. Para contornar isso, as estruturas são salvas em disco, mais especificamente em
um banco de dados não-relacional chamado Cassandra. Dessa forma, o Redis só terá as estruturas cujas respectivas 
discussões foram acessadas recentemente.

De modo simplificado, para cada nova visualização que precisar ser contada, precisamos verificar se a estrutura 
associada à discussão visualizada está presente no Redis. Se estiver, basta chamarmos o método~\texttt{PFADD} passando 
como argumentos o identificador da discussão como chave e o identificador do usuário, que pode ser o endereço IP, como
valor. E caso o retorno de \texttt{PFADD} seja 1, indicando que o contador foi incrementado, devemos atualizar a nova
contagem no Cassandra.

Se a estrutura~$\HLOG$ da discussão não existir no Redis, devemos procurá-la no Cassandra e criar uma nova caso ela não
exista. Em seguida, devemos carregá-la no Redis, e o modo de se fazer isso é chamar o método~\texttt{SET} passando como
parâmetro o identificador da discussão como chave e o $\HLOG$ como valor. Por fim, os mesmos passos da situação em que 
a discussão já está no Redis devem ser seguidos.

Por fim, devemos consultar o Cassandra para sabermos quantas visualizações distintas uma discussão específica teve.