%!TeX root=../tese.tex
%("dica" para o editor de texto: este arquivo é parte de um documento maior)
% para saber mais: https://tex.stackexchange.com/q/78101/183146

% Insira aqui os metadados do seu trabalho. Para isso, copie,
% com as alterações necessárias, o conteúdo do arquivo
% conteudo-exemplo/metadados.tex

\title{
    titlept={Um, dois, três, $\dots$ muitos, $\dots$},
    titleen={One, two, three, $\dots$ many, $\dots$},
    subtitlept={muitos mesmo},
    subtitleen={definitely many},
}

\author{William Hideki Kondo}
\orientador{José Coelho de Pina}

\defesa{
  nivel=tcc,
  programa={Ciência da Computação},
  local={São Paulo},
  data={2022-02-15},
}

\palavrachave{Contagem distinta}
\palavrachave{Aproximação}
\palavrachave{Fluxo de dados}

\keyword{Count-distinct}
\keyword{Approximation}
\keyword{Data stream}

\resumo{
\textbf{Contagem distinta} é o problema de se encontrar o número de elementos distintos em um fluxo de dados com 
repetição de elementos. A solução trivial, que insere os dados em um tabela, tem um consumo de espaço linear e é 
inviável para aplicações com alto volume de dados. Algoritmos probabilísticos resolvem esse problema trocando 
a exatidão da contagem por uma grande redução do consumo de espaço. Então, este texto apresentará soluções 
probabilísticas para a contagem distinta.
}

\abstract{
\textbf{Count-distinct} is the problem of finding the number of distinct elements in a data stream with repeated 
elements. The trivial solution, that inserts the data in a table, has a linear space consumption and it is impracticable
for high volume data applications. Probabilistic algorithms solve this problem by exchanging the exact count for a great 
space consumption reduction. Therefore, this text will introduce probabilistic solutions for count-distinct.
}
