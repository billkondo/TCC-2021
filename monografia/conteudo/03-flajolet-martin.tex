\newcommand{\bitmap}{\text{BITMAP}}
\newcommand{\pcounting}{\textit{probabilistic counting}}

\chapter{Contagem distinta aproximada: Flajolet-Martin (1985)}
\label{lab:flajolet-martin}

\section{O Problema}

\begin{quote}
  \textbf{Contagem Distinta:} Dado um conjunto $\Mbb$, \textbf{encontrar} quantos elementos \textit{distintos}
  $\Mbb$ possui.
\end{quote}

Uma solução para a \textbf{contagem distinta} é inserir cada elemento de $\Mbb$ em uma tabela hash. 
E assim, a quantidade de itens distintos nesse conjunto será o número de elementos nessa tabela.

Note que esse algoritmo consome $\Omega(n)$ de memória, uma vez que, deve-se armazenar cada elemento 
\textbf{pelo menos uma vez} para que seja possível verificar se um item está no conjunto.

Este consumo linear de espaço pode ser um problema quando a quantidade de elementos que precisam ser armazenados é muito 
grande. E podem existir aplicações nas quais é necessário manter \textbf{várias} dessas estruturas, como no caso do 
\textit{Redis}, em que se deseja encontrar quantos usuários distintos visualizararm um publicação ~\citep{Redis}. E para 
essas situações, é interessante que o consumo de memória seja muito menor que a quantidade de elementos distintos. 
Portanto, deve-se abandonar a exatidão da contagem dos elementos distintos para que se possa diminuir o consumo de 
espaço.

\begin{quote}
  \textbf{Contagem Distinta Aproximada:} dado um conjunto $\Mbb$, \textbf{estimar} o número de elementos 
  \textit{distintos} nesse conjunto.
\end{quote}

Uma das primeiras soluções para a \textbf{contagem distinta aproximada} foi apresentada por \textit{Philippe Flajolet} e 
\textit{Nigel Martin} no artigo ~\citep{flajolet:martin:85}.
A motivação para o desenvolvimento desse algoritmo foi a otimização de pesquisas em bancos de dados relacionais.
E a identificação do número de elementos distintos em uma coluna era a principal dificuldade nesse processo.

\section{Algoritmo de Flajolet-Martin}
\label{sec:flajolet-martin:algorithm}

O \textbf{algoritmo de Flajolet-Martin} utiliza uma função de hash $h$ que mapeia uniformemente cada elemento do 
conjunto $\Mbb$ para um número inteiro entre $0$ e $2^L-1$, em que $L$ é a quantidade necessária de bits para 
mapear todas as entradas de $\Mbb$ (na prática, este valor é $32$ ou $64$). Assim, para cada $x_i \in \Mbb$, 
será calculado um inteiro $y_i \coloneqq h(x_i)$. Note que $y_i$ pode ser visto como uma palavra binária aleatória de 
$L$ bits em que cada bit é gerado independentemente com probabilidade $\frac{1}{2}$. E o algoritmo é baseado na 
frequência das aparições dos prefixos dessas palavras aleatórias.

Então, suponha que palavras binárias aleatórias de $L$ bits estejam sendo geradas. Espera-se que a cada \textbf{2} 
palavras, pelo menos uma comece com \textbf{1}. E que a cada \textbf{4} palavras, pelo menos uma deva ter o prefixo 
\textbf{01}. Assim, a cada $\mathbf{2^k}$ palavras, espera-se que pelo menos uma palavra comece com $\mathbf{0^{k-1}1}$. 
Por outro lado, caso o prefixo $0^{k-1}1$ apareça, o esperado é que $2^{k}$ items já devam ter sido gerados. 

Dessa forma, o algoritmo mantém a ocorrência dos prefixos de cada $\textit{hash}$ em um vetor $\bitmap[0 \twodots L-1]$, 
de modo que $\bitmap[i] = 1$, se e somente se, o prefixo $0^i1$ apareceu entre os \textit{hashes} dos elementos de 
$\Mbb$. 

Note que é necessário mapear cada prefixo para um inteiro entre $0$ e $L - 1$. E para isso, defini-se a função $\rho$ 
que recebe um inteiro e devolve a posição do 1 menos significativo deste inteiro. Então, para um \textit{hash} $y$, 
suponha que $\rho(y) = k$. Assim, o prefixo da representação binária de $y$ é $0^k1$, e $\bitmap[k] = 1$. Um caso 
especial para essa função é quando o \textit{hash} é zero. Nesta situação, $\rho(0) = L$.

Por fim, seja $R$ o menor índice tal que $\bitmap[R] = 0$. A estimativa para o número de elementos distintos de 
$\Mbb$ será $2^R/\phi$, em que, $\phi$ é um fator de correção que será discutido em 
\refeq{sec:flajolet-martin:analysis}.

Segue o pseudocódigo:
\begin{programruledcaption}{
Algoritmo de Flajolet-Martin 
\\ \textbf{Entrada:} conjunto $\Mbb$, função de hash $h$, tamanho $L$ do vetor $\bitmap$ 
\\ \textbf{Saída:} quantidade de elementos distintos de $\Mbb$
\label{prog:flajolet-martin}
}
  \begin{lstlisting}[
    language={[brazilian]pseudocode},
    style=pseudocode,
    style=wider,
    functions={},
    specialidentifiers={},
  ]
      funcao FlajoletMartin($\Mbb$, h, L)
        para i de 0 até L \kw{faça}:
          $\bitmap[i]$ := 0
        fim
        para x $\in$ $\Mbb$ \kw{faça}:
          y := h(x)
          $\bitmap[\rho(y)]$ := 1
        fim
        R := 0
        enquanto $\bitmap[R]$ = 1 \kw{faça}:
          R := R + 1
        fim
        devolva $2^R/\phi$
      fim
  \end{lstlisting}
\end{programruledcaption}

\section{Análise do algoritmo}
\label{sec:flajolet-martin:analysis}

Suponha que a quantidade de elementos distintos de $\Mbb$ seja $n$. O valor esperado de $R$ definido na seção 
\refeq{sec:flajolet-martin:algorithm} é aproximadamente $\log_2(\phi n)$, em que $\phi = 0.77351\dots$ e o desvio padrão 
de $R$ é em torno de $1.12$. Essa seção busca mostrar as etapas da análise do Algoritmo~\ref{prog:flajolet-martin} feita 
por Flajolet e Martin. Os detalhes da prova podem ser vistos em ~\citep{flajolet:martin:85}.

O primeiro passo é entender que $R$ é uma estimativa para $\lfloor \log_2(n) \rfloor$. Pelo padrão das aparições dos 
prefixos dos hashes dos elementos de $\Mbb$, espera-se que $\bitmap[0]$ seja acessado aproximadamente $\lfloor 
\frac{n}{2} \rfloor$ vezes, $\bitmap[1]$ aproximadamente $\lfloor \frac{n}{4} \rfloor$ vezes $\dots$ Agora, para 
$k = \lfloor \log_2(n) \rfloor$, $\bitmap[k]$ deve ser acessado em torno de $\lfloor \frac{n}{2^{k+1}} \rfloor = 0$ 
vezes. Assim, o menor índice $R$ tal que o valor de $\bitmap$ seja zero é aproximadamente $\lfloor \log_2(n) \rfloor$.

Então, defini-se a variável aleatória $R_n$ como sendo o valor da variável $R$ ao final da execução do Algoritmo 
\ref{prog:flajolet-martin} para uma entrada $\Mbb$ com $n$ elementos distintos. Dessa forma, o interesse principal 
da demonstração é encontrar fórmulas ou estimativas para:
\begin{itemize}
  \item $p_{n,k} = \mathbb{P}(R_n = k)$: probabilidade de uma saída de \ref{prog:flajolet-martin} ser igual a $k$
  \item $q_{n,k} = \mathbb{P}(R_n \geq k)$: probabilidade de uma saída de \ref{prog:flajolet-martin} 
  ser maior ou igual a $k$
  \item $\Ebb[R_n]$: valor esperado de $R_n$
  \item $\Vbb[R_n]$: variância de $R_n$
\end{itemize}

O primeiro teorema de ~\citep{flajolet:martin:85} mostra uma fórmula \textit{exata} e \textit{discreta} para $q_{n,k}$. 
A principal idea para encontrar essa fórmula é agrupar as palavras binárias por prefixos da forma $0^k1$. Assim, 
defini-se $E_k = \{ x  \ | \ \rho(x) = k \}$, ou seja, $E_k$ é o conjunto de todas as palavras aleatórias com prefixos
iguais a $0^k1$. Da mesma forma, defini-se $K_k = \{ x \ | \ \rho(x) \geq k \}$. Em seguida, as diferentes entradas 
$\Mbb$ com $n$ elementos distintos são representadas por um polinômio:
\[ P_k^{(n)} = (E_0 + E_1 + \twodots + E_{k-1} + K_k)^n .\]

O próximo passo é tentar expandir esse polinômio usando \textit{inclusão e exclusão}, e associar uma medida 
probabilidade para $E_0, E_1, \twodots, E_{k-1}, K_k$. E a prova deste teorema termina encontrando uma relação entre 
$q_{n,k}$ e esta expansão de polinômio.

Em seguida, o Teorema 2 apresenta aproximações de $q_{n,k}$ para diferentes intervalos de $k$. E a consequência deste 
teorema é a existência de uma distribuição limitante para a distribuição de probabilidade de $R_n$ conforme $n$ cresce. 
Dessa forma, obtém-se uma fórmula \textit{aproximada} e \textit{contínua} para $q_{n,k}$:
\[ q_{n,k} \approx \psi(\frac{n}{2^k}) \]

em que, $\psi(x) = \prod_{j \geq 0} (1 - e^{-x2^j})$.

Note que por definição, $p_{n,k} = q_{n,k} - q_{n,k+1}$. Assim, pode-se aproximar $p_{n,k}$:
\[ p_{n,k} \approx \psi(\frac{n}{2^k}) - \psi(\frac{n}{2^{k+1}}) \ . \]

O interesse passa a ser, portanto, estimar $\Ebb[R_n]$ a partir dessa fórmula aproximada de $p_{n,k}$, de maneira 
que 
\[ \Ebb[R_n] = \sum_{k \geq 1} k p_{n,k} \approx \sum_{k \geq 1} k \Big[ \psi \Big( \frac{n}{2^k} \Big) - \psi 
  \Big( \frac{n}{2^{k+1}} \Big) \Big] \ . \]

Desse modo, defini-se a função real $F(x)$ como sendo
\[ F(x) =  \sum_{k \geq 1} k \Big[ \psi \Big( \frac{n}{2^k} \Big) - \psi \Big( \frac{n}{2^{k+1}} \Big) \Big] \ . \]

E o Lema 1 do artigo, afirma que 
\[ \Ebb[R_n] = F(x) + O \Big( \frac{1}{n^{0.49}} \Big) \ , \]
ou seja, que o valor esperado de $R_n$ se aproxima de $F(x)$ conforme $n$ cresce.

Em seguida, o Lema 2 apresenta o resultado da \hyperref[ap:mellin]{transformada de Mellin} de $F(X)$. A principal razão 
para se calcular essa transformação é que se pode expressar a fórmula inversa da transformada de Mellin como uma 
expansão assintótica cujos termos são resíduos da transformada. Assim, o Teorema $3A$ utiliza os Lemas 1 e 2, e o 
Teorema dos Resíduos para afirmar que 
\[ \Ebb[R_n] = \lg (\phi n) + P(\lg n) + o(1) \ , \]
em que, $P(x)$ é a expansão assintótica de $F(x)$ e $\phi = 0.77351\dots$, concluindo a prova que 
$\Ebb[R_n] \approx \lg (\phi n)$.

A prova para a estimativa de $\Vbb[R_n]$ segue os mesmos passos da prova anterior. Pela definição de 
\hyperref[ap:variance]{variância}, precisa-se estimar $\Ebb[R_n ^ 2]$. Assim, 
\[ \mathbb{E[R_n ^2]} = \sum_{k=1} k^2 p_{n,k} \approx G(x) \ , \]
em que
\[ G(x) = \sum_{k=1} k^2 p_{n,k} \ . \]

Dessa forma, encontra-se a transformada de Mellin de $G(x)$ e se analisa a inversa desta transformação para estimar 
$\Ebb[R_n^2]$.

\section{Melhorando precisão do algoritmo}

Na seção anterior, foi visto que $\Ebb[R_n] \approx \lg(\phi n)$, em que $\phi \approx 0.77351$. Assim, se o valor 
do contador $R$ no final do Algoritmo \ref{prog:flajolet-martin} para uma entrada com $n$ elementos distintos for 
aproximadamente igual a $\lg(\phi n)$, então a saída desse algoritmo é aproximadamente $n$. A tabela \ref{tab:flajolet} 
mostra que, em alguns casos, a saída do programa é praticamente igual a $n$ quando $R_n \approx \lg(\phi n)$.

\begin{center}
  \def\arraystretch{2}%
  \begin{table}
    \begin{tabular}{ |p{1.5cm}||p{2.5cm}|  }
      \hline
      \multicolumn{1}{|p{1.5cm}|}{\centering $n$ } 
      & \multicolumn{1}{|p{2.5cm}|}{\centering $(1 \slash \phi) 2^{\lg(\phi n)}$ }  \\
      \hline
      \multicolumn{1}{|p{1.5cm}|}{\centering 50 } 
      & \multicolumn{1}{|p{2.5cm}|}{\centering 49.99 }  \\
      \hline
      \multicolumn{1}{|p{1.5cm}|}{\centering 500 } 
      & \multicolumn{1}{|p{2.5cm}|}{\centering 500.0 }  \\
      \hline
      \multicolumn{1}{|p{1.5cm}|}{\centering 5000 } 
      & \multicolumn{1}{|p{2.5cm}|}{\centering 4999.99 }  \\
      \hline
      \multicolumn{1}{|p{1.5cm}|}{\centering 50000 } 
      & \multicolumn{1}{|p{2.5cm}|}{\centering 50000.0 }  \\
      \hline
     \end{tabular}
     \caption{\label{tab:flajolet} Comparação entre $n$ e saída do Algoritmo \ref{prog:flajolet-martin} para 
     $R_n = \lg(\phi n)$.}
  \end{table}
\end{center}

Contudo, $\Vbb[R_n] \approx\nolinebreak 1.12$ e consequentemente, o desvio padrão de $R_n$ é aproximadamente~$1$. 
Isto quer dizer que o valor de $R_n$ pode ser uma unidade maior ou menor que $\lg(\phi n)$, o que implica que a 
estimativa de $n$ possa ser duas vezes maior ou menor que $n$. Logo, o Algoritmo de Flajolet e Martin apresenta uma 
grande variabilidade.

Assim como foi visto na Seção \ref{sec:morris:plus}, pode-se diminuir a variância da estimativa fazendo $k$ iterações 
do Algoritmo \ref{prog:flajolet-martin}. Dessa forma, seria necessário manter $k$ vetores \bitmap, $k$ valores de $R$ e 
$k$ funções de hash $h$. Ao final de todas as $k$ iterações, seria calculado a média $\bar{R}$ dos valores de $R$, e a 
estimativa do número de elementos distintos seria $(1 \slash \phi) 2^{\bar{R}}$.

O pseudocódigo a seguir codifica essa idea:

\begin{programruledcaption}{
  Algoritmo de Flajolet-Martin+
  \\ \textbf{Entrada:} conjunto $\Mbb$, $k$ funções de hash, tamanho $L$ do vetor $\bitmap$ 
  \\ \textbf{Saída:} quantidade de elementos distintos de $\Mbb$
  \label{prog:flajolet-martin+}
  }
    \begin{lstlisting}[
      language={[brazilian]pseudocode},
      style=pseudocode,
      style=wider,
      functions={},
      specialidentifiers={},
    ]
        funcao FlajoletMartin($\Mbb$, k, h, L)
          para i de 0 até k \kw{faça}:
            para j de 0 até L \kw{faça}:
              $\bitmap_i[j]$ := 0
            fim
          fim
          para i de 0 até k \kw{faça}:
            para x $\in$ $\Mbb$ \kw{faça}:
              y := $h_i(x)$
              $\bitmap_i[\rho(y)]$ := 1
            fim
          fim
          para i de 0 até k \kw{faça}:
            $R_i$ := 0
          fim
          para i de 0 até k \kw{faça}:
            enquanto $\bitmap_i[R_i]$ = 1 \kw{faça}:
              $R_i$ := $R_i$ + 1
            fim
          fim
          $\bar{R}$ := 0
          para i de 0 até k \kw{faça}:
            $\bar{R}$ := $\bar{R} + R_i$ 
          fim
          $\bar{R}$ := $\bar{R} \slash k$
          devolva $2^{\bar{R}}/\phi$
        fim
    \end{lstlisting}
  \end{programruledcaption}

Seja $\bar{R_n}$ o valor da variável $\bar{R}$ ao final da execução do Algoritmo \ref{prog:flajolet-martin+} para uma
entrada com $n$ elementos distintos. De forma análoga aos resultados vistos na Seção \ref{sec:morris:plus}, pode-se
concluir que:
\[ \Ebb[\bar{R_n}] \approx \lg(\phi n)  \; \; \text{e}  \; \; \Vbb[\bar{R_n}] \approx \frac{1.12}{k} \ . \]


No entanto, o fato de se precisar de uma função de hash distinta para cada iteração torna a solução acima inviável, 
uma vez que, encontrar funções de hash que mapeiem na prática os elementos de um conjunto $\Mbb$ de maneira
uniforme não é uma tarefa simples. Para contornar esse problema, os autores propuseram o uso da 
\textit{média estocástica}.

Essa ideia consiste em dividr os elementos da entrada em $k$ lotes e usar parte da informação do hash para definir em 
qual lote um elemento deve ir. 
As linhas de $7$ até $11$ do pseudocódigo a seguir mostram como esta divisão é feita:
\begin{programruledcaption}{
  Algoritmo de Flajolet-Martin++
  \\ \textbf{Entrada:} conjunto $\Mbb$, funções de hash $h$, tamanho $L$ do vetor $\bitmap$, 
  quantidade $k$ de lotes
  \\ \textbf{Saída:} quantidade de elementos distintos de $\Mbb$
  \label{prog:flajolet-martin++}
  }
    \begin{lstlisting}[
      language={[brazilian]pseudocode},
      style=pseudocode,
      style=wider,
      functions={},
      specialidentifiers={},
    ]
        funcao FlajoletMartin($\Mbb$, k, h, L)
          para i de 0 até k \kw{faça}:
            para j de 0 até L \kw{faça}:
              $\bitmap_i[j]$ := 0
            fim
          fim
          para x $\in$ $\Mbb$ \kw{faça}:
            lote := $h(x) \mod k$
            y := $\lfloor h(x) / k \rfloor$
            $\bitmap^{<lote>}[\rho(y)]$ := 1
          fim
          para i de 0 até k \kw{faça}:
            $R_i$ := 0
          fim
          para i de 0 até k \kw{faça}:
            enquanto $\bitmap_i[R_i]$ = 1 \kw{faça}:
              $R_i$ := $R_i$ + 1
            fim
          fim
          $\bar{R}$ := 0
          para i de 0 até k \kw{faça}:
            $\bar{R}$ := $\bar{R} + R_i$ 
          fim
          $\bar{R}$ := $\bar{R} \slash k$
          devolva $k2^{\bar{R}}/\phi$
        fim
    \end{lstlisting}
  \end{programruledcaption}

Se a função de hash $h$ distribuir os $n$ elementos distintos do conjunto $\Mbb$ uniformemente entre os 
$k$ lotes, então espera-se que cada lote tenha aproximadament $\frac{n}{k}$ elementos. Dessa forma, 
$(1 / \phi)2^{\bar{R_n}}$ seria uma aproximação para $\frac{n}{k}$. É devido a este fato que a saída do Algoritmo~
\ref{prog:flajolet-martin++} é $\mathbf{k} (1 / \phi)2^{\bar{R_n}}$, pois este valor é uma estimativa para 
$\mathbf{k} \frac{n}{k} = n$.

