\chapter{Contagem distinta aproximada: Flajolet-Martin(1985)}
\label{lab:flajolet:martin}

\section{O Problema}

\begin{quote}
  \textbf{Contagem Distinta:} Dado um conjunto $\mathbb{M}$, \textbf{encontrar} quantos elementos \textit{distintos}
  $\mathbb{M}$ possui.
\end{quote}

Uma solução para a \textbf{contagem distinta} é inserir cada elemento de $\mathbb{M}$ em uma tabela hash. 
E assim, a quantidade de itens distintos nesse conjunto será o número de elementos nessa tabela.

Note que esse algoritmo consome $O(n)$ de memória, uma vez que, deve-se armazenar cada elemento 
\textbf{pelo menos uma vez} para que seja possível verificar se um item está no conjunto.

Este consumo linear de espaço pode ser um problema quando a quantidade de elementos que precisam ser armazenados é muito grande.
E podem existir aplicações nas quais é necessário manter \textbf{várias} dessas estruturas, como no caso do \textit{Redis}, em
que se deseja encontrar quantos usuários distintos visualizararm um publicação ~\citep{Redis}. E para essas situações, é interessante
que o consumo de memória seja muito menor que a quantidade de elementos distintos. Portanto, deve-se abandonar
a exatidão da contagem dos elementos distintos para se possa diminuir o consumo de espaço.

\section{Algoritmo de Flajolet-Martin}

\begin{quote}
  \textbf{Contagem Distinta Aproximada:} dado um conjunto $\mathbb{M}$, \textbf{estimar} o número de elementos \textit{distintos}
  nesse conjunto.
\end{quote}

Uma das primeiras soluções para a \textbf{contagem distinta aproximada} foi apresentada por \textit{Philippe Flajolet} e 
\textit{Nigel Martin} no artigo ~\citep{flajolet:martin:85}.
A motivação para o desenvolvimento desse algoritmo foi a otimização de pesquisas em bancos de dados relacionais.
E a identificação do número de elementos distintos em uma coluna era a principal dificuldade nesse processo.