\chapter{Contagem aproximada}
\label{chap:morris}

\section{Problema}

O problema de contagem aproximada foi abordado pela primeira vez por Robert  Morris (Morris, 1978). 
Neste artigo, o autor descreve a tentativa de se contar eventos cujas frequências podiam chegar até 130.000, mas só usando contadores de 8 bits.

Um registrador de n bits pode guardar valores até $2^n-1$. Dessa forma, em uma máquina que possui registradores de 8 bits, pode-se contar até 255.
Assim, o autor não conseguia manter as frequências exatas dos eventos devido à limitação de máquina. 
Contudo, ele podia armazenar contagens aproximadas.

\section{Ideias}

Apresentar ideias para atacar o problema \dots

\section{Pseudocódigo}

Escrever pseudocódigo \dots

\section{Prova}

Escrever a prova do algorimo \dots
