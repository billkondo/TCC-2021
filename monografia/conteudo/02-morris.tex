\chapter{Contagem aproximada}
\label{chap:morris}


\section{O Problema}

O problema de contagem aproximada consiste em contar um grande número de eventos usando pouca memória.  
Esse problema foi abordado pela primeira vez por Robert  Morris no artigo ~\citep{morris:78}, 
em que o autor descreve a tentativa de se contar eventos cujas frequências podiam chegar até 130.000, mas só usando contadores de 8 bits.

Um registrador de n bits pode guardar valores até $2^n-1$. Dessa forma, em uma máquina que possui registradores de 8 bits, pode-se contar até 255.
Assim, o autor não conseguia manter as frequências exatas dos eventos devido à limitação de máquina. 
Contudo, ele podia armazenar contagens aproximadas.


\section{Ideias para solução}

Para se manter um contador exato até $n$, precisa-se de $O(\log n)$ bits. Para se conseguir contar até $n$ usando menos bits, 
deve-se abrir mão da exatidão da contagem. 

Uma das primeiras ideias é manter no contador o valor de $\log_2 n$ e assim, utilizar $O(\log \log n)$ bits de memória. 
A estimativa da contagem seria $2^x$, em que, $x$ é o valor armazenado no contador.

Outra ideia é como deve ser feita o incremento desse contador. 
A solução proposta por Morris é aumentar o contador com base em um método probabilístico, como pode ser visto no Programa \ref{prog:morris}. 


\section{Pseudocódigo}

Morris propôs originalmente manter um contador $X$ que poderia ser aumentado com probabilidade $2^{-X}$ a cada novo item que precisasse ser contado.
Assim, a estimativa do total de itens seria $2^{X} - 1$, sendo que o $-1$ é para que essa aproximação funcione para quando não há itens, ou seja, $X = 0$.

No entanto, o pseudocódigo abaixo é também baseado em uma ideia da seção \textit{Algorithm} do artigo ~\citetitle*{ApproximateCountingAlgorithm} (~\cite{ApproximateCountingAlgorithm}), 
em que, é exposta a noção de expressar a condição de incremento do contador como um experimento de lançamento de moedas. 
Se um novo item precisa ser contado e o contador guarda o valor $X$, então será feito o lançamento de $X$ moedas. Se todas as moedas forem cara, então o contadador é incrementado.
E a probabilidade deste evento ocorrer é $2^{-X}$. 

Segue o pseudocódigo:
\begin{programruledcaption}{Contagem aproximada: algoritmo de Morris\label{prog:morris}}
  \begin{lstlisting}[
    language={[brazilian]pseudocode},
    style=pseudocode,
    style=wider,
    functions={},
    specialidentifiers={},
  ]
      funcao Morris(M)  // Estima o tamanho de um conjunto de dados M
        X := 0  // Inicia o contador que guarda o logarítmo na base 2 do tamanho do conjunto M
        para cada dado do conjunto M faça
          jogue uma moeda X vezes

          se X = 0 ou todas as jogadas forem cara faça
            X := X + 1
          fim
        fim
      devolva $2^X - 1$
      fim
  \end{lstlisting}
\end{programruledcaption}

\section{Análise do Algoritmo}

Esta seção busca esclarecer o quanto a estimativa do Programa \ref{prog:morris} se distância do tamanho real do conjunto estimado.
Considere que $X_n$ é a saída deste programa para uma entrada de tamanho $n$.

\begin{lemma}
A estimativa do Programa \ref{prog:morris} é não-viesada, ou seja, $\mathbb{E}[2^{X_n} - 1] = n$.
\end{lemma}

\begin{proof}
A prova será por indução em $n$. 

\textbf{Caso base:} Para $n = 0$, $X_n$ = 0. Logo, $\mathbb{E}[2^0 - 1] = 0$

\textbf{Passo indutivo:} Suponha que $n > 0$ e que $\mathbb{E}[2^{X_{n-1}} - 1] = n-1$.

\begin{align*}
  \mathbb{E}[2^{X_n} - 1] &= \sum_{x = 0} \mathbb{P} (X_{n-1} = x) \ \mathbb{E}[2^{X_n} - 1 | X_{n-1} = x] \\
                          &= \sum_{x = 0} \mathbb{P} (X_{n-1} = x) \ ((2^{x+1} - 1) \frac{1}{2^x} +  (2^x - 1) (1 - \frac{1}{2^x})) \\
                          &= \sum_{x = 0} \mathbb{P} (X_{n-1} = x) \ 2^x \\
                          &= \sum_{x = 0} \mathbb{P} (X_{n-1} = x) \ (2^x - 1) + \sum_{x = 0} \mathbb{P} (X_{n-1} = x) \\
                          &= \mathbb{E}[2^{X_n-1} - 1] + 1 \\
                          &= n - 1 + 1 \\
                          &= n
\end{align*}

Serão feitos alguns comentários sobre a equação acima. Primeiro, por se tratar de uma prova por indução em $n$, é interessante tentar escrever 
$\mathbb{E}[2^{X_n} - 1]$ em função de $\mathbb{E}[2^{X_n-1} - 1]$. Portanto, aplica-se a definição de valor esperado e escreve-se $\mathbb{E}[2^{X_n} - 1]$
em função de $X_{n-1}$, de modo que $\mathbb{E}[2^{X_n} - 1] = \sum_{x = 0} \mathbb{P} (X_{n-1} = x) \ \mathbb{E}[2^{X_n} - 1 | X_{n-1} = x]$ . 

Para calcular $\mathbb{E}[2^{X_n} - 1]$ dado que $X_{n-1} = x$, basta considerar dois casos: 
\begin{itemize}
  \item o contador é aumentado com probabilidade $2^{-x}$ e a nova estimativa de $n$ é $2^{x+1} - 1$.
  \item o contador não é aumentado com probabilidade $1 - 2^{-x}$ e a estimativa de $n$ permanece como $2^{x} - 1$.
\end{itemize}

Assim, $\mathbb{E}[2^{X_n} - 1 | X_{n-1} = x] = (2^{x+1} - 1) \frac{1}{2^x} +  (2^x - 1) (1 - \frac{1}{2^x})$. 
A partir deste ponto, basta manipular as contas para concluir que $\mathbb{E}[2^{X_n} - 1] = n$.

\end{proof}

