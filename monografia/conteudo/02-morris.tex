\chapter{Contagem aproximada}
\label{chap:morris}


\section{O Problema}

O problema de contagem aproximada consiste em contar um grande número de eventos usando pouca memória.  
Esse problema foi abordado pela primeira vez por Robert  Morris (Morris, 1978). 
Neste artigo, o autor descreve a tentativa de se contar eventos cujas frequências podiam chegar até 130.000, mas só usando contadores de 8 bits.

Um registrador de n bits pode guardar valores até $2^n-1$. Dessa forma, em uma máquina que possui registradores de 8 bits, pode-se contar até 255.
Assim, o autor não conseguia manter as frequências exatas dos eventos devido à limitação de máquina. 
Contudo, ele podia armazenar contagens aproximadas.


\section{Ideias para solução}

Para se manter um contador exato até $n$, precisa-se de $O(\log n)$ bits. Para se conseguir contar até $n$ usando menos bits, 
deve-se abrir mão da exatidão da contagem. 

Uma das primeiras ideias é manter no contador o valor de $\log_2 n$ e assim, utilizar $O(\log \log n)$ bits de memória. 
A estimativa da contagem seria $2^x$, em que, $x$ é o valor armazenado no contador.

Outra ideia é como deve ser feita o incremento desse contador. 
A solução proposta por Morris é aumentar o contador com base em um método probabilístico, como pode ser visto no Programa \ref{prog:morris}. 


\section{Pseudocódigo}

\begin{programruledcaption}{Contagem aproximada: algoritmo de Morris\label{prog:morris}}
  \begin{lstlisting}[
    language={[brazilian]pseudocode},
    style=pseudocode,
    style=wider,
    functions={},
    specialidentifiers={},
  ]
      funcao Morris(M)  // Estima o tamanho de um conjunto de dados M
        X := 0  // Inicia o contador que guarda o logarítmo na base 2 do tamanho do conjunto M
        para cada dado do conjunto M faça
          para i de 0 até X faça
            jogue uma moeda 
          fim

          se todas as jogadas forem cara faça
            X := X + 1
          fim
        fim
      devolva $2^X - 1$
      fim
  \end{lstlisting}
\end{programruledcaption}

\section{Erro cometido}

Escrever a prova do algorimo \dots
